\section{Rendering}
\label{sec:rendering}

We use an approach similar to Koerner et al.~\cite{Koerner14}, where we seperate the radiance field into single scattered light $L_{ss}$ and multiple scattered light $L_{ms}$:
\begin{align}
L\left(\vec{x},\omega\right) = L_{ss}\left(\vec{x},\omega\right) + L_{ms}\left(\vec{x},\omega\right)
\ .
\end{align}

The single scattered light is folded into the emission term $Q$:
\begin{align}
Q(\vec{x}, \omega) = L_{ss}(\vec{x}, \omega) = \sigma_s\left(\vec{x}\right)\int_\Omega{ p\left(\omega'\rightarrow\omega\right) L_{u}\left(\vec{x}, \omega'\right)\ud\omega' }
\ .
\end{align}

This means that our solver will solve for the multiple scattered light $L_{ms}$. The quantity $L_u$ is the uncollided light, which was emitted from the light source and attenuated by the volume without undergoing any scattering event.
We compute it using a few light samples per voxel, which quickly converge to a useful result for Dirac delta light sources.

Running the solver gives solution vector $\vec{u}$. We then un-stagger the solution by interpolating all coefficients to voxel centers. The additional coefficients at boundary voxels are no longer needed. This operation is represented as a matrix that produces a three-dimensional voxel grid with SH coefficients for order $N$ at the center of each voxel.

For rendering, we use a simple forward path tracing approach, where we start tracing from the camera. At the first scattering event, we use next event estimation to account for $L_{ss}$. Then we sample a new direction according to the phase function. Instead of continuing tracing into the new direction, we evaluate the in-scattering integral using $\widehat{L}_{ms}$. The SH coefficients at $\vec{x}$ are found by trilinear interpolation from the voxel grid of SH coefficients.


