% ---------------------------------------------------------------------
% EG author guidelines plus sample file for EG publication using LaTeX2e input
% D.Fellner, v2.02, Jan 25, 2017


%\title[EG \LaTeX\ Author Guidelines]%
%{\LaTeX\ Author Guidelines for EUROGRAPHICS Proceedings Manuscripts}

% for anonymous conference submission please enter your SUBMISSION ID
% instead of the author's name (and leave the affiliation blank) !!
%\author[D. Fellner \& S. Behnke]
%{\parbox{\textwidth}{\centering D.\,W. Fellner\thanks{Chairman Eurographics Publications Board}$^{1,2}$
%        and S. Behnke$^{2}$ 
%        S. Spencer$^2$\thanks{Chairman Siggraph Publications Board}
%        }
%        \\
%% For Computer Graphics Forum: Please use the abbreviation of your first name.
%{\parbox{\textwidth}{\centering $^1$TU Darmstadt \& Fraunhofer IGD, Germany\\
%         $^2$Graz University of Technology, Institute of Computer Graphics and Knowledge Visualization, Austria
%%        $^2$ Another Department to illustrate the use in papers from authors
%%             with different affiliations
%       }
%}
%}


% ------------------------------------------------------------------------

% if the Editors-in-Chief have given you the data, you may uncomment
% the following five lines and insert it here
%
% \volume{36}   % the volume in which the issue will be published;
% \issue{1}     % the issue number of the publication
% \pStartPage{1}      % set starting page


%-------------------------------------------------------------------------
\begin{document}

% uncomment for using teaser
% \teaser{
%  \includegraphics[width=\linewidth]{eg_new}
%  \centering
%   \caption{New EG Logo}
% \label{fig:teaser}
%}

%\maketitle

%-------------------------------------------------------------------------

%\begin{abstract}
%TODO

%\begin{classification} % according to http://www.acm.org/class/1998/
%\CCScat{Computer Graphics}{I.3.7}{Three-Dimensional Graphics and Realism}{Raytracing}
%\end{classification}

%\end{abstract}


%-------------------------------------------------------------------------
\section{Introduction}

Simulating light transport in paticipating media is still a challenging problem for image synthesis in computer graphics. Monte Carlo based techniques have established as the standard approach, because they can produce unbiased results and are conceptionally easy to implement. They become computationally demanding however, for participating media, which exhibits strong scattering or anisotropy.

Deterministic methods on the other hand are unpopular. This is because they suffer from discretization artifacts, produce biased results, cannot be coupled easily with surface rendering problems and are trickier to implement. However, their appeal lies in the fact that they have better performance for certain problems and produce a global solution across the whole domain.

The work on path guiding techniques from recent years has shown, how discretized approximate representations of the correct solution can be used to accelerate Monte Carlo integration techniques, such as path tracing. Instead of generating these approximate representations using correct methods, deterministic methods may offer a viable alternative and pave the way for hybrid methods, which combine the performance benefits of deterministic methods with correct and exact Monte Carlo techniques. Further, deterministic methods lend themselfs to applications, where a fast approximate solution is often preferred over correct, but slowly converging results. Under this light, it is important to study deterministic methods and have a solid understanding about their characteristics and performance traits for typical rendering problems.

The $P_N$-method is a deterministic method, which is popular in other fields like medical imaging and nuclear sciences, but has not been explored for applications in graphics yet. The purpose and main contribution of our paper is, to gain a solid understanding of its foundations and present a method for application in rendering. In particular, we present the following theoretical and practical contributions:
\begin{itemize}
	\item We derive and present the time-independent real-valued $P_N$-equations.
	\item We introduce a staggered-grid solver, for which we generate stencil code automatically from a computer algebra representation of the $P_N$-equations. This allows us to deal with the increasingly complex equations, which the $P_N$-method produces for higher order. It further allows our solver to be used for any (potentially coupled) partial differential equation, which results in a system of linear equations after discretization.
	\item Finally, we compare the $P_N$-method for higher orders against flux-limited diffusion, and answer the long standing question about which approach should be preferred.
\end{itemize}

In the next section, we will discuss related work and why it motivates our contribution. Section~\ref{sec:pnequations} introduces the real-valued $P_N$-equations, which we solve using our solver described in section~\ref{sec:pnsolver}. We explain in section~\ref{sec:rendering}, how the solution is being applied in a rendering application. Our $P_N$-solver is run and compared against flux-limited diffusion for a set of standard problems in section~\ref{sec:results}. We finish with our conclusion in section~\ref{sec:conclusion}.

%-------------------------------------------------------------------------
\section{Previous work}

\subsection*{Deterministic Methods}

\begin{itemize}
  \item Lattice Boltzmann Methods \cite{Geist04}
  \item Discrete Ordinates \cite{Languenou95}
  \item $PN$-Method \cite{Kajiya84} 
    \begin{itemize}
        \item Kajia briefly explained the theory, but did not give any details on implementation or how to solve it. In fact, as Max~\cite{Max95} points out, it is not clear if he succeeded at all, as all of Kajia's results were produced with a simpler method. 
        \item We decided to use the $P_N$-method, because it has not been studied and applied in the context of computer graphics before.
    \end{itemize}
  \item Diffusion \cite{Stam95} \cite{Jakob10}
  \item Variable Eddington Factors (VEF) \cite{Koerner14}
  \item Heuristics \cite{Kaplanyan10} \cite{Elek14}
\end{itemize}

\subsection*{Path-guiding}

\begin{itemize}
  \item Russian roulette and splitting \cite{Vorba16}
  \item Learning using SD-Trees \cite{Pegoraro08} \cite{Mueller17} or Gaussian mixture models \cite{Vorba14}
  \item Zero-Variance sampling \cite{Krivanek14}
  \item Reinforcement learning \cite{Dahm17}
\end{itemize}




%-------------------------------------------------------------------------
\section{Radiative Transfer and $PN$-Equations}

In order to compute a rendered image, the radiance field $L$ is evaluated. Its evolution throught space is given by the radiative transfer equation (RTE):

\begin{itemize}
  \item give RTE
  \item also introduce operator notation (used later when talking about least-squares form of the RTE)
\end{itemize}


The radiance field $L$ is typically solved with probabilistic Monte-Carlo methods, such as path-tracing. With these methods, computational demand is directly linked to the average length of paths and the presence of light transport paths which are hard to find with random sampling. The path length increases for dense, highly scattering media and hard-to-sample paths exist with highly anisotropic phase functions or complex visbility in the scene.

Deterministic methods are derived by discretizing the radiance field $L$ and the RTE in spatial- and spherical domain. The result is a system of linear equations of the form $A\vec{x}=b$ where $\vec{x}$ represents the radiance field in the discretized problem. The matrix $A$ and right hand side $b$ are found by assembling a global system from all discretization elements. Deterministic methods suffer from discretization artefacts and only give an accurate solution in the discretization limit. However, they are fast to compute and noise free.

The $PN$-equations are found by projecting the radiance field and the RTE into a truncated spherical harmonics (SH). A regular grid is used for spatial discretization. The SH-expansion of the radiance field $L$ is:

\begin{equation}
L\left(x, \omega \right) = \sum_{l=0}^{N}
\sum_{m=-l}^{l}
L\left(x\right)^{lm}
Y\left(\omega\right)^{l,m}
\end{equation}

where $N$ is the truncation order, $L^{lm}$ is a set SH coefficients and $Y^{l,m}$ is the SH-basis function.



\section{System Overview}
\section{Stencil Generation}
\begin{itemize}
  \item Manually generating the stencil code for higher orders is intractable due to the increasing number of coefficient equations. We therefore automatically generate the stencil code directly from a computer algebra representation of the expanded RTE.
  \item Arbitrary order. Staggered and collocated grids. Neumann and dirichlet boundary conditions. Generalized to arbitrary coupled PDE's. Could be extended to other problem domains, such as shallow water simulation.
\end{itemize}


\section{System Solving}

\begin{itemize}
  \item Staggered grids are required in order to get rid of oscillating artefacts in the $P_N$-solution.
  \item CDA produces a non-symmetric diagonal-dominant matrix A, which allows the application of Gauss-Seidel method.
  \item PN-equations produce a non-symmetric, non-diagonal dominant matrix A, which cannot be solved with standard methods.
  \item Bringing $A\vec{x}=\vec{b}$ into normal form $A^TA\vec{x}=A^T\vec{b}$, produces a symmetric, positive definit system matrix $A^TA$, which allows application of standard methods, such as Conjugate-Gradient.
  \item Solving the normal form is equivalent to solving the least-squares form of the RTE equation. (maybe give it in operator notation).
  \item The system matrix of the normal form has a higher condition number than the standard form.
\end{itemize}

\subsection*{CDA vs. $P_1$}

\begin{itemize}
  \item CDA is a degenerated form of $P_1$. It is derived by isolating the flux-vector on one side of the vector-equation formed by the $l=1$ SH-band equations. This isolation requires division by the extinction coefficient, introducing a $\frac{1}{\sigma_t}$ factor which diverges as $\sigma_t$ approaches zero. Thresholding to some minimum is required, in order to be able to solve the system for vacuum regions.
  \item $P_1$ does not require any thresholding of $\sigma_t$ as it does not contain $\sigma_t$ as a denominator. It therefore can deal with vacuum regions without modifications.
  \item (needs validation) Further, in the presence of vacuum or near vacuum regions, the condition number for CDA is higher than for $P_1$, because of small extinction values in the denominator of the diffusion coefficient.
  \item Using the normal form for CDA will further increase the condition number when vacuum regions are present and significantly decreases convergence.
\end{itemize}

\section{Path-guiding}
\section{Results}
\section{Conclusion and future work}

\begin{itemize}
  \item Extension to surface rendering by dealing vacuum regions and extending to anisotropic RTE.
  \item Apply our automated stencil generation to other problem domains where coupled PDE's are present.
\end{itemize}

%-------------------------------------------------------------------------

\bibliographystyle{eg-alpha}
%\bibliographystyle{eg-alpha-doi}

\bibliography{bibliography}

%-------------------------------------------------------------------------
%\newpage



\end{document}
