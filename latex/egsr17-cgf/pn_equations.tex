\section{Real-valued $P_N$-Equations}
\label{sec:pnequations}

In this section we present how to optain the real-valued $P_N$-equations. For brevity and readability, we only give a brief outline here and refer to the supplemental material for the detailed derivation.

The $P_N$-equations are derived from the radiative transfer equation, which expresses the change of the radiance field $L$, with respect to an infinitesimal change of position into direction $\omega$ at point $\vec{x}$:
\begin{align}
%\label{eq:rte}
\left(\nabla\cdot\omega\right)L\left(\vec{x}, \omega \right)
=&
-\sigma_t\left(\vec{x}\right) L\left(\vec{x}, \omega \right)\nonumber\\
&
+\sigma_s\left(\vec{x}\right) \int_{\Omega}
{
p\left(\omega'\cdot\omega\right)L\left(\vec{x}, \omega \right)\ud\omega'
}\nonumber\\
&
+Q\left(\vec{x}, \omega\right)\nonumber
\end{align}

The left hand side (LHS) is the transport term and we refer to the terms on the right hand side (RHS) as collision, scattering and source term respectively. The symbols $\sigma_t$, $\sigma_s$, $p$ and $Q$ refer to the extinction coefficient, scattering coefficient, phase function and emission term.

The RTE is often given in operator notation, where transport, collision and scattering are expressed as operators $\mathcal{T}$, $\mathcal{C}$ and $\mathcal{S}$, which are applied to the radiance field $L$:
\begin{align}
\mathcal{T}\left(L\right) = -\mathcal{C}\left(L\right) + \mathcal{S}\left(L\right) + Q
\end{align}

The $P_N$-method expands the angular dependence into spherical harmonics. Since the radiance field $L$ is real only, we use the real-valued SH basis functions $\SHBR^{l,m}$, which are defined in terms of the complex-valued SH basis functions $\SHBC^{l,m}$:
\begin{align*}
\SHBR^{l,m}=
\left\{
\begin{array}{lr}
\frac{\iu}{\sqrt{2}}\left(\SHBC^{l,m}-\left(-1\right)^m\SHBC^{l,-m}\right), & \text{for } m < 0\\
\SHBC^{l,m}, & \text{for } m = 0\\
\frac{1}{\sqrt{2}}\left(\SHBC^{l,-m}-\left(-1\right)^m\SHBC^{l,m}\right), & \text{for } m > 0
\end{array}
\right.
\end{align*}

We express the projection into the SH basis functions with the projection operator $\mathcal{P}$:
\begin{align}
\mathcal{P}^{l, m}(f) = \int_{\Omega}f(\mathbf{x}, \omega) \SHBR^{l,m}(\omega)\,\mathrm{d}\omega = f^{l,m}
\nonumber
\end{align}

The $P_N$-equations are derived by first replacing the radiance field $L$ by its SH reconstruction $\widehat{L}$, which introduces an error due to truncation at order $N$:
\begin{align*}
\widehat{L}\left(\vec{x}, \omega\right) =
\sum_{l=0}^{N}
{
\sum_{m=-l}^{l}
{
L^{l,m}\left(\vec{x}\right)\SHBR^{l,m}\left(\omega\right)
}
}
\approx
L\left(\vec{x}, \omega\right)
\end{align*}

After similarly replacing all angular dependent RTE parameters, we project each term into SH, using the projection operator $\mathcal{P}$. This produces a single equation for each $l,m$-pair. The $P_N$-equations therefore can be written as:
\begin{align}
\mathcal{P}^{l,m}\mathcal{T}\left(\widehat{L}\right)
=
-\mathcal{P}^{l,m}\mathcal{C}\left(\widehat{L}\right) 
+\mathcal{P}^{l,m}\mathcal{S}\left(\widehat{L}\right)
+\mathcal{P}^{l,m}\left(Q\right)
\nonumber
\end{align}

Expanding the operators and writing the real-valued $P_N$-equations in compact form, requires a string of expansions, manipulations and applications of identities. We here give the final result and refer to the supplemental material for the detailed derivation. Since the real-valued SH basis have different definitions, depending on $m<0$, $m=0$ or $m>0$, we get different projections $\mathcal{S}^{l,m}$ depending no the sign of $m$.

The real-valued $P_N$-equations are for $m<0$:
\begin{align}
&-\frac{1}{2}c^{\scaleto{l-1,m-1}{4pt}}
\partial_y
L^{\scaleto{l-1,-m+1}{4pt}}
%\\
+\frac{1}{2}d^{\scaleto{l+1,m-1}{4pt}}
\partial_y
L^{\scaleto{l+1,-m+1}{4pt}}
%\\
-\frac{1}{2}\beta^{\scaleto{m}{4pt}}e^{\scaleto{l-1,m+1}{4pt}}
\partial_y
L^{\scaleto{l-1,-m-1}{4pt}}
\nonumber
\\&
+\frac{1}{2}\beta^{\scaleto{m}{4pt}}f^{\scaleto{l+1,m+1}{4pt}}
\partial_y
L^{\scaleto{l+1,-m-1}{4pt}}
%\\
+\frac{1}{2}\delta_{\scaleto{m\neq -1}{4pt}}c^{\scaleto{l-1,m-1}{4pt}}
\partial_x
L^{\scaleto{l-1,m-1}{4pt}}
\nonumber
\\&
-\frac{1}{2}\delta_{\scaleto{m\neq -1}{4pt}}e^{{l-1,m+1}}
\partial_x
L^{\scaleto{l-1,m+1}{4pt}}
%\\
+\frac{1}{2}f^{\scaleto{l+1,m+1}{4pt}}
\partial_x
L^{\scaleto{l+1,m+1}{4pt}}
%\\
-\frac{1}{2}d^{\scaleto{l+1,m-1}{4pt}}
\partial_x
L^{\scaleto{l+1,m-1}{4pt}}
\nonumber
\\&
+a^{\scaleto{l-1,m}{4pt}}
\partial_z
L^{\scaleto{l-1,m}{4pt}}
%\\
+b^{\scaleto{l+1,m}{4pt}}
\partial_z
L^{\scaleto{l+1,m}{4pt}}
%\\
+\sigma_t L^{\scaleto{l,m}{4pt}}
%\\
-\sigma_s\lambda_{\scaleto{l}{4pt}}p^{\scaleto{l,0}{4pt}}L^{\scaleto{l,m}{4pt}}
%\\
= Q^{\scaleto{l,m}{4pt}}
\label{eq:rpn_m_<_z}
\end{align}

for $m=0$:
\begin{align}
&
\frac{1}{\sqrt{2}}c^{\scaleto{l-1,-1}{4pt}}\partial_x L^{\scaleto{l-1,1}{4pt}}
-\frac{1}{\sqrt{2}}d^{\scaleto{l+1,-1}{4pt}}\partial_x L^{\scaleto{l+1,1}{4pt}}
%\\&
\frac{1}{\sqrt{2}}c^{\scaleto{l-1,-1}{4pt}}\partial_y L^{\scaleto{l-1,-1}{4pt}}
\nonumber
\\&
-\frac{1}{\sqrt{2}}d^{\scaleto{l+1,-1}{4pt}}\partial_y L^{\scaleto{l+1,-1}{4pt}}
%\\&
a^{\scaleto{l-1,0}{4pt}}\partial_z L^{\scaleto{l-1,0}{4pt}}
+b^{\scaleto{l+1,0}{4pt}}\partial_z L^{\scaleto{l+1,0}{4pt}}
\nonumber
\\&
+\sigma_t L^{\scaleto{l,m}{4pt}}
-\sigma_s\lambda_{\scaleto{l}{4pt}}p^{\scaleto{l,0}{4pt}}L^{\scaleto{l,m}{4pt}}
= Q^{\scaleto{l,m}{4pt}}
\label{eq:rpn_m_=_z}
\end{align}

and for $m>0$:
\begin{align}
&
\frac{1}{2}c^{\scaleto{l-1,-m-1}{4pt}}\partial_x L^{\scaleto{l-1,m+1}{4pt}}
%\\&
-\frac{1}{2}d^{\scaleto{l+1,-m-1}{4pt}}\partial_x L^{\scaleto{l+1,m+1}{4pt}}
%\\&
-\frac{1}{2}\beta^{\scaleto{m}{4pt}}e^{\scaleto{l-1,m-1}{4pt}}\partial_x L^{\scaleto{l-1,m-1}{4pt}}
\nonumber
\\&
+\frac{1}{2}\beta^{\scaleto{m}{4pt}}f^{l+1,-m+1}\partial_x L^{\scaleto{l+1,m-1}{4pt}}
%\\&
+\frac{1}{2}c^{\scaleto{l-1,-m-1}{4pt}}\partial_y L^{\scaleto{l-1,-m-1}{4pt}}
\nonumber
\\&
-\frac{1}{2}d^{\scaleto{l+1,-m-1}{4pt}}\partial_y L^{\scaleto{l+1,-m-1}{4pt}}
%\nonumber
%\\&
+\delta_{\scaleto{m\neq 1}{4pt}}\frac{1}{2}e^{\scaleto{l-1,-m+1}{4pt}}\partial_y L^{\scaleto{l-1,-m+1}{4pt}}
\nonumber
\\&
-\delta_{\scaleto{m\neq 1}{4pt}}\frac{1}{2}f^{\scaleto{l+1,-m+1}{4pt}}\partial_y L^{\scaleto{l+1,-m+1}{4pt}}
%\\&
+a^{\scaleto{l-1,-m}{4pt}}\partial_z L^{\scaleto{l-1,m}{4pt}}
%\nonumber
%\\&
+b^{\scaleto{l+1,-m}{4pt}}\partial_z L^{\scaleto{l+1,m}{4pt}}
\nonumber
\\&
+\sigma_t L^{\scaleto{l,m}{4pt}}
-\sigma_s\lambda_{\scaleto{l}{4pt}}p^{\scaleto{l,0}{4pt}}L^{\scaleto{l,m}{4pt}}
= Q^{\scaleto{l,m}{4pt}}
%\nonumber
\label{eq:rpn_m_>_z}
\end{align}

with
\begin{align*}
\label{eq:real_sh_basis}
\beta^{x}=
\left\{
\begin{array}{lr}
\frac{2}{\sqrt{2}}, & \text{for } \vert x\vert = 1\\
1, & \text{for } \vert x\vert \neq 1
\end{array}
\right.
,\quad
\delta_{x\neq y}=
\left\{
\begin{array}{lr}
1, & \text{for } x \neq y \\
0, & \text{for } x = y
\end{array}
\right.
\end{align*}

and
\begin{align*}
&
a^{\scaleto{l,m}{4pt}}= \sqrt{\frac{\left(l-m+1\right)\left(l+m+1\right)}{\left(2l+1\right)\left(2l-1\right)}} \qquad
b^{\scaleto{l,m}{4pt}}= \sqrt{\frac{\left(l-m\right)\left(l+m\right)}{\left(2l+1\right)\left(2l-1\right)}}
\\&
c^{\scaleto{l,m}{4pt}}= \sqrt{\frac{\left(l+m+1\right)\left(l+m+2\right)}{\left(2l+3\right)\left(2l+1\right)}} \qquad
d^{\scaleto{l,m}{4pt}}= \sqrt{\frac{\left(l-m\right)\left(l-m-1\right)}{\left(2l+1\right)\left(2l-1\right)}}
\\&
e^{\scaleto{l,m}{4pt}}= \sqrt{\frac{\left(l-m+1\right)\left(l-m+2\right)}{\left(2l+3\right)\left(2l+1\right)}} \qquad
f^{\scaleto{l,m}{4pt}}= \sqrt{\frac{\left(l+m\right)\left(l+m-1\right)}{\left(2l+1\right)\left(2l-1\right)}}
\end{align*}
\begin{align*}
\lambda_l=\sqrt{\frac{4\pi}{2l+1}}
\end{align*}

\begin{figure}[h]
\centering
\missingfigure{test}
\vspace{-0.2in}
\icaption{Structure of coefficient matrix $A$ and solution vector $\vec{u}$ after discretization of the $P_N$-equations on a finite difference grid.}
\end{figure}

\begin{figure*}[t]
\centering
\missingfigure{$P_N$-solver overview: generate stencil code $\rightarrow$ build system $\rightarrow$ solve $\rightarrow$ render}
\vspace{-0.2in}
\icaption{Overview of our $P_N$-solver. After generating the stencil source code from the expression trees representing the $P_N$-equations, the linear system $A\vec{u}=\vec{Q}$ is built using RTE parameter fields and additional user input, such as grid resolution and type of boundary conditions. The resulting system is solved for $\vec{u}$, which is then used in our rendering application.}
\label{fig:pnsolver}
\end{figure*}

The spatial variable $\vec{x}$ is discretized using a finite difference (FD) voxelgrid. The radiance field $L$, after full discretization, is represented as a set of coefficients per voxel. Flattening these over all voxels into a single vector gives the solution vector $\vec{u}$. The RHS vector $\vec{Q}$ is produced similarly. The differential operators are discretized using central differences. This allows the projected operators to be expressed as linear transformations, which can be collapsed into a single coefficient matrix $A$:
\begin{align}
(T+C-S)\vec{u} = A\vec{u} = \vec{Q}
\end{align}

$T$, $C$, $S$ are matrices, which result from the discretized transport, collision and scattering operators in the $P_N$-equations respectively.



In the next section, we will present our solver for solving the real-valued $P_N$-equations. 

