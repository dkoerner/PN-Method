\section{Rendering}
\label{sec:rendering}

In order to evaluate our results, we chose to use to render the solution directly. We use an approach similar to Koerner et al.~\cite{Koerner14}, where we seperate the radiance field into single scattered light $L_{ss}$ and multiple scattered light $L_{ms}$:
\begin{align}
L\left(\vec{x},\omega\right) = L_{ss}\left(\vec{x},\omega\right) + L_{ms}\left(\vec{x},\omega\right)
\end{align}

The single scattered light is folded into the emission term $Q$:
\begin{align}
Q(\vec{x}, \omega) = L_{ss}(\vec{x}, \omega) = \sigma_s\left(\vec{x}\right)\int_\Omega{ p\left(\omega'\rightarrow\omega\right) L_{u}\left(\vec{x}, \omega'\right)\ud\omega' }
\end{align}

This means that, our solver will solve for the multiple scattered light $L_{ms}$. The quantity $L_u$ is the uncollided light, which was emitted from the lightsource and attenuated by the volume without undergoing any scattering event. It is easily computed for delta light sources, such as directional lights, in homogeneous volumes. Raymarching can be used in heterogeneous volumes. For more complex light sources, the emission field can be precomputed with a number of Monte Carlo iterations. 

Running the solver gives solution vector $\vec{u}$. The staggering is removed by interpolating all coefficients to voxel centers. Also the additional coefficients at boundary voxels are removed. This operation is represented as a matrix and results in a simple three-dimensional voxelgrid with SH coefficients for order $N$ at the center of each voxel.

For rendering, we use a simple forward path tracing approach, where we start tracing from the camera. At the first scattering event, we use next event estimation to account for $L_{ss}$. Then we sample a new direction according to the phase function. Instead of continuing tracing into the new direction, we simply evaluate the inscattering integral using $\widehat{L}_{ms}$. The SH coefficients at $\vec{x}$ are found by trilinear interpolation from the voxelgrid of SH coefficients.


